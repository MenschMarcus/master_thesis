%Hinführung mit Relevanz, Lösungsskizzen und Vorgehen (Gliederung)
%Notwendige Vorkenntnisse über Walbdbarnd und Notfallsituationen sowie jetziges Vorgehen.

\subsection{Katastrophenmanagement}

\cite{yuan_intelligent_2005} nennen als Aufgaben eines Katastrophenmanagementsystems: Überwachen, Berichten, Identifizieren, Hinweisen, Organisieren, Operieren, Beurteilen und Untersuchen.

In \cite{tomaszewski_geovisual_2007} wird die Entwicklung von Katastrophenmanagement mit interaktiv visualisierten georeferenzierte Daten untersucht. Eine geovisuelle Analyse hilft, um Situationen wahr zu nehmen, Probleme zu lösen und Entscheidungen zu treffen. \citep{thomas_illuminating_2005} betonen, dass die Konzepte geovisueller Analyse-Systeme allgemeinen Regeln der visuellen Analyse folgen: Verwendung von Wahrnehmungsgrundlagen bei Techniken für die Bewältigung schwieriger Aufgaben; Beachtung des Verhältnisses zwischen Komplexität und Dringlichkeit; Ver\-bindung der Informationen von verschiedenen Quellen zu eindeutig dargestellten Bedeutungen. Dem stimmen \citeauthor{tomaszewski_geovisual_2007} zu.

Eine weitere wichtige Informationsquelle für das Katastrophenmanagement sind vergangene oder mögliche Katastrophen-Szenarien. Visualisierungen helfen bei der Entwicklung von Übungsszenarien, dem Auswerten vergangener Entscheidungen und dem frühen Erkennen von Schwachstellen. Ein gutes Beispiel hierfür ist die Mustererkennung bei Verkehrsstaus. Die Fähigkeit dynamisch graphische Komponenten einer Schnittstelle zu manipulieren, ermöglicht eine Analyse komplexer Ereignisse. \citep{tomaszewski_geovisual_2007}