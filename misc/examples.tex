% references
\footnote{
  \textit{Title},
  Author,
  URL: \url{http://},
  last access:
}
                    % (for online resources)
\cite{source_id}    % in the text itself
\ref{label_name}    % for reference other parts of the thesis

% bib
@BOOK {key,
  title = {{name}}
  publisher = {name, ISBN}
  year = {YYYY}
  author = {name1 and name2 and name3}
}

@ARTICLE{key,
  title = {{name}},
  journal = {name},
  volume = {number},
  pages = {number--number},
  month = {MMM},
  year = {YYYY},
  author = {name1 and name2 ...},
}

% for anything else
@MISC{key,
  title       = {\emph{name}},
  author      = {{name}},
  lastchecked = {DD MMM YYYY},
  url         = {url}
}

% figure with graphic
\begin{figure}[ht]  % places it more or less here
    \begin{center}
        \includegraphics[width=350px, clip=true, trim=0 0 0 0]{graphics/name_of_file}
    \end{center}
    \caption{description of graphic}
    \label{fig:graphics_name}
\end{figure}

% table (working example)
\begin{table}[ht]
\centering
\begin{tabular}{llp{1em}ll}
    \toprule
    \multicolumn{2}{c}{klassische Karte} & & \multicolumn{2}{c}{moderne Karte} \\
    Charakter & Beschränkung & & Charakter & Aufhebung \\
    \midrule
    statisch & nur diskrete Zustände & & dynamisch & kontinuierliche Prozesse \\
    isolierend & nur ein Teil des Georaums & & allumfassend & mehrere Detailstufen \\
    selektiv & nur eine Darstellungsform & & inklusiv & mehrere Perspektiven \\
    passiv & nur Sender von Informationen & & interaktiv & Manipulation der Darstellung \\
    \bottomrule
\end{tabular}
\caption{Aufhebung der Beschränkungen der klassischen Kartografie}
\label{tab:kartografie}
{\small Abwandlung nach \cite{karcher}}
\end{table}


% manipulation of vertical space
\vspace{-1em}       % on one page
\newpage            % go to next page

% list and descriptions
\begin{description}[labelindent=1.53em] % manipulation of indentation
    \item[$^1$]
    \item[$^2$]
    \item[$^3$]
\end{description}

% quotes
\begin{quoteit}
\large
This is my quote!
\end{quoteit}
\hfill \textit{-- and this is where it came from)}

% Math
$A \neq B$          % inline example

% source code
\begin{figure}[H]
\begin{lstlisting} [numberfirstline=true,numbers=left,stepnumber=1]
  ... code ...
\end{lstlisting}
\caption*{description_of_file}
\label{code_label_of_file}
\end{figure}
