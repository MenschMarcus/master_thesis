%!TEX root = ../thesis.tex


If some countries have too much history, we have too much geography. Mackenzie King
Read more at: http://www.brainyquote.com/search_results.html?q=history+country

\chapter{Introduction}
\label{cha:introduction}



This thesis deals exactly with that topic: How can an information system be designed that is able to gather, manage and present historical changes of countries in the course of history? To be more precise, the thesis will deal with the following research questions:

\begin{enumerate}
  \item How to represent the temporal changes of countries, their borders and their names in a geographic information system?
  \item How to visualize these historical changes on the map?
  \item How to deal with uncertainty, imprecision and debated territories?
\end{enumerate}

The work will first lay a foundation with an overview of the working of a GIS and the analysis of existing approaches of spatio-temporal knowledge representation. Afterwards, an own concept to overcome shortcomings of the existent approaches will be developed. This concept will be implemented using \textsc{HistoGlobe}, an existing Web-based visualization of the historical evolution of countries, which will extent the software to a full HGIS. Finally, the concept will be evaluated in a user study and its implications can be used to determine future work to be done in the field of Historical geographic information systems.

\subsection{Motivation} % (fold)
\label{sub:motivation}

% subsection motivation (end)


\subsection{History} % (fold)
\label{sub:history}

% subsection history (end)


\subsection{Geography} % (fold)
\label{sub:geography}

% subsection geography (end)


\subsection{Research Questions} % (fold)
\label{sub:research_questions}

% subsection research_questions (end)


% chapter introduction (end)

system already works in browsing mode, i.e. experience history:
http://histoglobe.com
(Europe 1871-1990)

problem: very little data, data entry into the system is horribly complicated, there is no backend

solution: an editor for historical data.

problem: it has to cope with data that is depended on time (how long is the country active?) and space (what is the territorial extent ,i.e. the border, of the country?) -> spatio-temporal data

solution: application of spatio-temporal data models

problem: there are a lot of them

solution: choose the best

problem: there is no ``best''. Each and every one has advantages and disadvantages

solution: alright, then take the one with the biggest advantages.

comparison of two models: snapshot model (SM) vs. event-based spatio-temporal data model (ESTDM)

SM: save the status of the world at time point t\_1 (e.g. 1945) and time point t\_2 (e.g. 1990)
(+) very easy
(+) concept well known from historical maps
(+) very helpful if the status at the years directly is to be seen (e.g. 1945 or 1990)
(-) what if I want to know how the world looked like at t\_x (e.g. 1975)
=> it does not work, because there is no information
(-) every time something changes on the map a total new image would have to be saved
=> very redundant and inefficient

ESTDM: save the status of the world at one reference time point t\_r (e.g. 1945) and from there on save only changes to this one, e.g.
1949: separation of Occupied Germany into East Germany and West Germany and ceding of parts of the country to Poland
1990: unification of West Germany and East Germany to Germany
(-) a little bit more complex
(+) for each point in history the status of the world can be retrieved: it is the status at the reference time point t\_r and all changes until this point accumulated and applied on the map
(+) historical changes can easily be visualized on the map, because they are stored explicitly
(-) storing a change backwards (e.g. 1933: rename of Weimar Republic into Nazi Germany) is more complex, because it has to be applied in the opposite direction

there are some more problems I dealt with. But finally I have developed a prototype that can do that (click\_prototype.pdf) that can do it. This prototype was developed with the aid of the people in Scholar's Lab that have been a big part of my development process.

event locations are sometimes not related to their consequences (e.g. Conferences of Tehran or Casablanca (1943) discussed how to deal with Germany after planned victory in WWII)

