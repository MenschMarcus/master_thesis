%!TEX root = ../thesis.tex

% ------------------ Literature ------------------ %
Langran 1989
Peuquet 1994
Wachowitz 1999
Yuan 1999
Peuquet 2002

% http://link.springer.com/article/10.1007/s12145-009-0027-6/fulltext.html
% http://stackoverflow.com/questions/5863676/table-design-for-spatio-temporal-data


% ------------------ TODO: ------------------ %

Multidimensional Geographic Information Science (Raper, 2001)
\cite{raper2000multidimensional}
Spatio-Temporal Narratives (Solana 2014) - chapter 2

whole book:
\cite{Langran1989timeingis}

chapters:
1 Fuzzs Sets ...
Spatio-Temporal Databases (Caluwe et al, 2010)
-> ontology of imperfection
\cite{deCaluwe:2010:SDF:1965517}

ott swiaczny
Time-Integrative Geographic Information Systems
pp  chapter   topic
2   into      triadic model
53  2.5       spatio-temporal dimensions
105 4.4       spatio-temporal GIS approaches
137 5.3       temporal queries  (temporal queries.pdf)


\subsection{Analysis} % (fold)
\label{sub:analysis}

% TODO: spatio-temporal queries

analysis: determine, change, evaluate

analysis methods
  spatial analysis
  temporal analysis
    time series analysis
    process analysis      (modification modeling + future forecasting)
  attribute analysis
alteration of \cite[p. 128]{ott2001time}

spatial queries
  query of spatial properties and attribute values
  e.g. size of Germany in 1871
thematic queries
  query objects based on certain criteria (spatial and attribute)
  multi-criteria analysis
  e.g. all democratic countries larger than 10.000 km²
statistical analysis
  artithmetic calculation and classification of characteristics of objects
  univariate, multivariate investigation
  e.g. What was the population density in Germany in 1945 compared to 1995
overlay/split
  aggregation and splitting of spatial components based on the layer principle
  e.g. Germany in 1945 gets split up into FRG and GDR by one polyline (inner German border) and one polygon (West Berlin)
geometric-topological operations
  analyze the neighborhood relations between geometric objects
  e.g. is the geometry all countries at time point 1991 strictly connected?
temporal analysis
  using spatial and temporal operators in figure
  %\ref{fig:spatial_temporal_operators}
  e.g. in which year did the largest amount of border changes happen?
\cite[p. 129-140]{ott2001time}

Multivariate Historical-Geographical Model
  multivariate
    features of a spatial object
    connection between temporal development of features
  geographical model
    location (geometry)
    neighborhood relation (topology)
  historical model
    object at different points in times
  \cite[p. 128]{ott2001time}
  % \cite Kilchenmann 1992

Spatial Queries / Operators
  and     intersection
  not     difference
  or      (cascaded) union
  xor     (inverted) symmetric difference

Temporal Queries / Operators

temporal logic
  rules and symbols
  represent time
  reason about time
  temporal operators
  ??? detail?
  % \cite Hodkinson and Reynolds 2006

  trajectories
    sequence of 2D or 3D locations of an object

Spatio-Temporal Queries / Operators
  when + where -> what
  when + what -> where
  where + what -> when
  % \cite Peuquet 1994


% ------------------------------------------------------------------------------
A GIS shall solve the problem for a user or answer his or her research question. Given a well-filled database with a working DBMS, the data might not answer the research question directly. It has to be sorted, selected or classified in order to convey the required information. For this process there are \emph{spatial operations} on the data in the system. Several operations can be applied in a certain order
\cite[pp. 321-325]{bolstad2008gis}.
Both spatial and attribute data are analyzed to combine the dimensions \emph{where?} and \emph{what?} in order to answer the ultimate question \emph{why?} something is the way it is
\cite[p.xii-xvi]{knowles2002past}.
