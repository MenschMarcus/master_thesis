%!TEX root = ../thesis.tex

\chapter{Introduction}
\label{cha:introduction}
From everything we know by today, planet Earth is the only place in our universe that is habitable for us. About 200,000 years ago the \emph{homo sapiens sapiens}, the first modern human beings, settled in nowadays Africa. 125,000 years ago humans conquered fire and the advent of agriculture 10,000 years ago started permanent human settlements in villages and introduced the end of nomadic human life. The settlements were often close to natural water, at rivers, lakes or the coast. Water is the most essential element -- and knowing where it is one of the most essential tasks for early human beings. Finding the habitat of animals was crucial for hunters, the location of bushes and shrubs for gatherers. Knowing the closest lookout was helpful to observe herd moving or get the latest weather forecast. Since the early days of mankind, knowing \emph{where} things are was essential for survival or at least helpful in the daily lives.

There is one tool that has been used already in early civilizations -- and it is one of the most beautiful and representative documents of these times: A map. It is a physical expression of something that is not tangible: The structure, the surface, the population of the Earth's surface. The old map that we know of was dates 4,000 years back, to the time of the Akkadian Dynasty of Sargon in Mesopotamia, currently Iraq. Ever since maps became important, may it be for navigators on their explorations on sea, in the field of land-use planning to design our human settlements or for rescue services on their mission to save the lives of people in danger. Nowadays, these tasks are fulfilled using geographic information systems (GIS) -- with a map being their most common type of presentation.

However, most of GIS answer two basic questions about an inspected object: \emph{Where} the object is in relative or absolute location and \emph{what} it is, being its attributes or properties. As an example, a city will have an exact geographic location, expressed in coordinates. Additionally there can be meta-information about the place: Its name, its population or the current rate of unemployment.

However, most of the GIS that are used nowadays focus mostly on spatial information and are limited to answer these two question. But what about the dimension of time? The answer to the question \emph{when} a city was found, how its population or city borders have developed over the previous fifty years or at what point we can expect it to reach a new threshold of population? To answer these questions, most of the GIS currently in use are not suitable and task-specific systems have to be used. The reason is that the temporal information is not supported most by GIS. But it is exactly the dimension of time that is so strongly related to our lives. It is the core aspect of history -- the study of our past, to understand the present and reason about the future.

Time and space are often tied, just as the French geographer Élisée Reclus described it in his study about human history on Earth. A connection of time and space in an \textbf{Historical geographic information system} \emph{(HGIS)} has a great potential to teach, learn and understand processes in the past. A system that is able to tell \emph{what} happened and \emph{where} the historical event has influences on and \emph{when} the event happened happened might be tool to answer the most important of all questions: \textbf{\emph{Why}} it happened?

This thesis deals exactly with that topic: How can an information system be designed that is able to gather, manage and present historical changes of countries in the course of history? To be more precise, the thesis will deal with the following research questions:

\begin{enumerate}
  \item How to represent the temporal changes of countries, their borders and their names in a geographic information system?
  \item How to visualize these historical changes on the map?
  \item How to deal with uncertainty, imprecision and debated territories?
\end{enumerate}

The work will first lay a foundation with an overview of the working of a GIS and the analysis of existing approaches of spatio-temporal knowledge representation. Afterwards, an own concept to overcome shortcomings of the existent approaches will be developed. This concept will be implemented using \textsc{HistoGlobe}, an existing Web-based visualization of the historical evolution of countries, which will extent the software to a full HGIS. Finally, the concept will be evaluated in a user study and its implications can be used to determine future work to be done in the field of Historical geographic information systems.

\subsection{Motivation} % (fold)
\label{sub:motivation}

% subsection motivation (end)


\subsection{History} % (fold)
\label{sub:history}

% subsection history (end)


\subsection{Geography} % (fold)
\label{sub:geography}

% subsection geography (end)


\subsection{Research Questions} % (fold)
\label{sub:research_questions}

% subsection research_questions (end)


% chapter introduction (end)


visualizing the course of history on a map and a timeline
-> setting a date on the timeline, seeing the status of history at this point on the map and emphasizing the changes since the last time

status = names and borders of countries
changes = change of names and borders of countries

system already works in browsing mode, i.e. experience history:
http://histoglobe.com
(Europe 1871-1990)

problem: very little data, data entry into the system is horribly complicated, there is no backend

solution: an editor for historical data.

problem: it has to cope with data that is depended on time (how long is the country active?) and space (what is the territorial extent ,i.e. the border, of the country?) -> spatio-temporal data

solution: application of spatio-temporal data models

problem: there are a lot of them

solution: choose the best

problem: there is no ``best''. Each and every one has advantages and disadvantages

solution: alright, then take the one with the biggest advantages.

comparison of two models: snapshot model (SM) vs. event-based spatio-temporal data model (ESTDM)

SM: save the status of the world at time point t\_1 (e.g. 1945) and time point t\_2 (e.g. 1990)
(+) very easy
(+) concept well known from historical maps
(+) very helpful if the status at the years directly is to be seen (e.g. 1945 or 1990)
(-) what if I want to know how the world looked like at t\_x (e.g. 1975)
=> it does not work, because there is no information
(-) every time something changes on the map a total new image would have to be saved
=> very redundant and inefficient

ESTDM: save the status of the world at one reference time point t\_r (e.g. 1945) and from there on save only changes to this one, e.g.
1949: separation of Occupied Germany into East Germany and West Germany and ceding of parts of the country to Poland
1990: unification of West Germany and East Germany to Germany
(-) a little bit more complex
(+) for each point in history the status of the world can be retrieved: it is the status at the reference time point t\_r and all changes until this point accumulated and applied on the map
(+) historical changes can easily be visualized on the map, because they are stored explicitly
(-) storing a change backwards (e.g. 1933: rename of Weimar Republic into Nazi Germany) is more complex, because it has to be applied in the opposite direction

there are some more problems I dealt with. But finally I have developed a prototype that can do that (click\_prototype.pdf) that can do it. This prototype was developed with the aid of the people in Scholar's Lab that have been a big part of my development process.
