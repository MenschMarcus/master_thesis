%!TEX root = ../masters_thesis.tex

\section{Basics} % (fold)
\label{sec:basics}

This chapter will lay the theoretical foundation of this Master's Thesis and will embed it into the context of current research. The title of this work is:

\vspace{-1em}
\begin{center}
\textbf{\titleFirst \\ \titleSecond}
\end{center}

It includes the domain (\emph{history of countries}) and the system to acquire, model, manage and visualize data of the domain: \emph{Historical Geographic Information Systems} (HGIS).

The first section of this chapter will explain the terms related to HGIS, including the domain. Afterwards, concepts to model time and space in an information system are introduced. Data sources suitable for input into an HGIS are listed in the next part, followed by techniques to manage and analyse the data. A special focus lies on concepts to visualize spatial and temporal data, explained in the next section. The chapter closes with possible HGIS applications and introduces the tool that is used in this thesis: \hg.


% ==============================================================================

% ==============================================================================

% ==============================================================================

% ==============================================================================

% ==============================================================================



%%%%%%%%%%%%%%%%%%%%%%%%%%%%%%%%%% CURRENT %%%%%%%%%%%%%%%%%%%%%%%%%%%%%%%%%%%%%


Related Terms
  Country
  History
          (research, Event)
  Geography
  Information
          (sign -> Data -> Information -> Knowledge)
  System

          (data in system spatial relation to Earth and temporal relation)
          (comparison: geo vs. his)

Models
        (Model: real world ---abstraction---> model)
        (dimensions: where, when, what -> triadic framework)
  Spatial Data Model
    Types of Space
        (PPG ->...-> PT)
    Geospatial Coordinates
    Geodetic Datum
    Vector vs. Pixel Graphics
    Geospatial Topology
    Spatial Operators
        (Boolean Operators, Set Operations)
  Temporal Data Model
    Types of Time
        (event vs. process)
        (valid / world time)
    Taxonomic Model of Time
    Temporal Topology
        (temporal relationships)
    Temporal Operators
  Spatio-Temporal Data Models
        (developments Driven By Changes
        Continuous Changes Vs. Discrete Changes
        Discrete: Idea Of State Machine (nothing Changes Until Event Happens)
        Continuous: Object Always Changes According To A Continuous Function)
    Spatio-Temporal Data Type
    Space-Time Cube
    Space-based Approaches
      Snapshot Model
      Time Slices
      The Grid Model
      Space-Time Composite
      Amendment Vector Model
    Time-based Approaches
      Time-stamping Model Topology Of Time
      Event-based Spatio-Temporal Data Model
      Object-Oriented Spatio-Temporal Data Model
      Cell-tuple-based Spatio-Temporal Data Model

Input
        (sources)

Management
  Spatio-Temporal Databases
    Database Management Systems
    Version Management
    Spatio-Temporal Queries
  Historical R-tree
  MV3R Tree

Analysis
  Multivariate Historical-Geographical Model
  Spatial Queries
  Temporal Queries
  Spatio-Temporal Queries

Presentation
  Scivis vs. Infovis
  Spatial Presentation
    Maps
  Temporal Presentation
    Timelines

Application
  Digital Humanities
  HistoGlobe --> TOOL OF THIS THESIS