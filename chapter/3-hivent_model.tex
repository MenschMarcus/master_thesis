%!TEX root = ../masters_thesis.tex

\chapter{The Hivent Model} % (fold)
\label{cha:the_hivent_model}

The task of this thesis is how to model, visualize and edit the development of countries in time and space in a HGIS. Before the system itself can be developed, the underlying data model has to be fixed. This is the purpose of this chapter.

In section \ref{sec:spatio_temporal_data_models}, different spatio-temporal data models were introduced to solve this problem. The \emph{Snapshot Model} was seen as unsuitable for the problem space. \emph{Simple Time-Stamping} is helpful to link countries to their history, but the method does not explicitly model historical changes, which is desireable. For that purpose, the idea of the \emph{Event-Based Spatio-Temporal Data Model} was developed, but since it only works for raster data, it is also not suitable for this thesis. This problem is solved in the \emph{History Graph Model}. Additionally, the introduced temporal changes allow to represent historical changes and their influences on geographic entities directly in the model. Finally, the \emph{Three-Domain Model} introduces a helpful concept to separate the spatial, temporal and thematic dimension of a spatio-temporal entity. The \emph{Hivent Model} developed in this thesis is constructed from components of some of these models: An Event-Based Spatio-Temporal Data Model organized according to the Three-Domain Model, supporting vector data and visualization on a History Graph.

In the first section of this chapter, the main elements of the Hivent model are introduced. Afterwards, the preconditions are defined. The next section explains the Hivent operations that are used to change countries in time and space. The data structures used to organize the elements of the model are illustrated in the third section.
% The chapter closes with an informal proof that the model is approproate.

% ==============================================================================
\section{Elements} % (fold)
\label{sec:elements}

The main elements of the the model are the \emph{Hivent}, representing an historically significant happening, an \emph{Area}, an abstract entity on a map with a name and a territory and an \emph{Historical Change}, that changes the history of one or more Areas and belongs to one Hivent.

% - - - - - - - - - - - - - - - - - - - - - - - - - - - - - - - - - - - - - - -
\paragraph{Hivents} % (fold)
\label{par:hivent}

are the main organizing element of the data model. The word is an acronym for \emph{\textbf{Hi}}storical e\emph{\textbf{vent}}. It represents a significant happening in history at one specific time point, e.g. a treaty, bill or declaration. In this domain, the focus is on events that influence the geopolitical situation Earth. That means, they introduce historical changes at their point in history. An Hivent has four different attributes:

\begin{compactenum}
  \item The \emph{name} of the Hivent
  \item The point in time, identified by the Hivent \emph{date}.
  \item A textual description of the Hivent \emph{location}.
  \item The \emph{historical changes} resulting from the Hivent.
\end{compactenum}

% paragraph hivent (end)

% - - - - - - - - - - - - - - - - - - - - - - - - - - - - - - - - - - - - - - -
\paragraph{Areas} % (fold)
\label{par:area}

are the the visible entities on the map. They are an abstract representation of one identical current or historical country. The model can easily be extended to model the history of states, provinces or regions. Therefore from now on the term \emph{political unit}, not \texttt{country} is used for the element in the real world. Each area has a \emph{short name}, e.g. ``Germany'', and a \emph{formal name}, e.g. ``Federal Republic of Germany''. The main spatial dimension of the data model is the Areas \emph{territory} represented by a polypolygon. The polypolygon is a set of weakly simple polygons, because the model has to support enclaves and exclaves. The polylines of the polygons represent the borders of the political unit. One border has always either one or two neighboring Areas and two neighboring Areas share a discrete set of common borders. A polyline consists of an ordered set of points, representing the border points.

An Area can change over time. Throughout the lifetime of an Area, it is created at some point, then its territory and short name can change any number of times and at some point it ceases. Since all changes in this model are sudden, there are only two possible states an Area can be in:
\begin{compactenum}
  \item An Area is \emph{active}, if at the current time point it is historically existing, with a name and a territory associated to it.
  \item On a contrarty, if an Area does not historically exist at this time point, it is \emph{inactive}.
\end{compactenum}

Each area is uniquely identified by its formal name. That means, the short name can change, but as soon as the formal name of an area changes (e.g. ``German Empire'' to ``Federal Republic of Germany''), it is considered a ``new'' Area.

% TODO: graphics  - - - - - - - O __________________ O - - - - - - - -
%                  inactive    t_s      active      t_e     inactive

% TODO: countries with enclaves or islands are not topologically equivalent.

% paragraph area (end)

% ------------------------------------------------------------------------------
\paragraph{Historical Changes} % (fold)
\label{par:historical_changes}

drive the development of an Area over time. They can create new Areas, update their territory and name and cease them. Each Historical Change belongs to one Hivent, inheriting its time point at which the change of the Areas happens. The actual historical change is defined by a set of Hivent Operations that will be introduced in section \ref{sec:hivent_operations}.

% paragraph historical_changes (end)

% section elements (end)

% ==============================================================================
\section{Preconditions} % (fold)
\label{sec:preconditions}

\begin{quoteit}
In the beginning God created the heavens and the earth \\
Now the earth was formless and empty [...] \\
And God said, “Let there be light” --- and there was light.
\end{quoteit}
\hfill -- Genesis 1:1, The First Book of Moses, Old Testament

This section defines the exact problem space and assumptions that the Hivent Model is based on. As defined in sections \ref{sub:model_of_geographical_space} and \ref{sub:presentation_of_geographic_space}, the surface of the Earth is curved, but can be projected on the two-dimensional map. Each point on the surface can uniquely be described by a pair of coordinates: latitude and longitude.

\newtheorem{invariant_surface}[assumptioncounter]{Assumption}
\begin{invariant_surface}
\label{ass:invariant_surface}
The surface of the Earth has an invariant area, i.e. it does not change over time. It is called the \emph{universe} entity and is denoted as $\Omega$.
\end{invariant_surface}

\newtheorem{initial_configuration}[assumptioncounter]{Assumption}
\begin{initial_configuration}
\label{ass:initial_configuration}
At the inital state of the system, $\Omega$ covers the whole surface of the Earth.
\end{initial_configuration}

\newtheorem{change}[assumptioncounter]{Assumption}
\begin{change}
\label{ass:change}
Each change of an Area is the result of exactly one Historical Change that belongs to exactly one Hivent.
\end{change}

The area name changes according to sudden events, e.g. a declaration or a governmental bill. The territory of a political entity can change either because of a geographical processes, e.g. the change of the coastline, or according to a historical event, e.g. a treaty. The model in this thesis focuses only on discrete historical changes and not on long-term geographical developments. It is assumed that the geographical conditions on Earth, especially the position of land and water and the coastlines have not changed in history. While this assumption is obviously wrong, it helps to keep the problem space clear. The data model will be open to future extensions to account also for geographic changes. In this data model, the temporal behavior of an Area can therefore be described as a \emph{static object that changes according to sudden events}.



Assumption \ref{ass:invariant_surface} is the fundamental axoim of the data model: $\Omega$ is used as the universe entity and it initially covers



Earth is initially fully covered with one area: $\Omega$
1) secession of water $W$ from $\Omega$
    => rest = usable (unclaimed) land
    this part can change due to model of long-term continuous geographic changes
2) cede rest of Areas from $\Omega$
  countries
  debated territories
  unknown land
  water

The data model only represents sudden changes of Areas, no processes, i.e. changes with duration. The model also assumes that coastlines never changed. Additionally to these two constraints, it is assumed that the surface of the Earth is divided into two types surfaces: \emph{water} and habitable \emph{land}. Land can at any point in time be either \emph{claimed}, i.e. it is currently the territory of exactly one active Area, or on a contrary be \emph{unclaimed}.

base: Newtons concept of absolute space?

=> topological rule:
each border has exactly two neighboring Areas
each Area has at least one neighboring Area

abstraction of NCH to ACH (attribute change) => all 7 cases:
ICH
UNI INC
SEP SEC
BCH ACH

BCH can geographically also be reduced to combination of separation and unification, but that would historically be different

no support for hierarchies

no support for independent overlapping areas

=> one layer

% section preconditions (end)

% ==============================================================================
\section{Hivent Operations} % (fold)
\label{sec:hivent_operations}

% section hivent_operations (end)



% ==============================================================================

\vspace{2em}
transition to next chapter

% chapter the_hivent_model (end)