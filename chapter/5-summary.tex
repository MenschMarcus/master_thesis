%!TEX root = ../masters_thesis.tex

\chapter{Summary} % (fold)
\label{cha:summary}


This Master's thesis started with the motivation to lay the foundation for an Historical Geographic Information System that shows the history of countries on Earth. While there are many interesting visualizations about historical topics, wars or events, there is no such thing as an interactive historical world atlas. This may be due to the fact that there is no comprehensive collection of historical data and that the whole nature of history is that everything we know is potentially uncertain. Even the commonly accepted concept of a ``country" is impossible to define without running into conflicts. To create an information system in such a complex domain is very challenging.


% ==============================================================================
\section{Results} % (fold)
\label{sec:results}

To summarize the most important results and contributions of this Master's thesis, the four research questions are finally answered.

\begin{description}[labelindent=0.55em]
  \item[\textbf{1)}]
  \textbf{
    What type of historical changes can happen in the development of countries in time and space?
  }
\end{description}

The history of countries is very complex. The problem space of this work was limited to the territoes and the names of countries. Except for the coastlines, the more interesting interior borders have always changed due to sudden events. The same is true for the name of countries. If a universe $\Omega$ is defined as an ever-existing territory that initially covered the whole surface of the Earth, then this thesis has shown an interesting result: With the exception of the rare case of reincarnation, eventhing that has ever happened to names and territories of countries can be expressed by five basic historical changes: Unification, separation, incorporation, secession and name change.

\begin{description}[labelindent=0.1em]
  \item[\textbf{2a)}]
  \textbf{
    How can these changes be modeled in an information system?
  }
\end{description}


These five changes are modelled as \emph{Hivent Operations}. Countries are represented by abstract \emph{Areas} with a formal name, a short name and a territory. Areas can change due to \emph{Hivents} -- \emph{\textbf{Hi}storical e\texttt{vents}} -- that introduce the Hivent Operations to cease or create a set of Areas or to update the properties of an Area over time.


\begin{description}[labelindent=0.1em]
  \item[\textbf{2b)}]
  \textbf{
    How can these changes be edited by humans in a user interface?
  }
\end{description}

While the Hivent Operations are


edit operations well understood by a human
1:n
hivent operatoins wlel understood by the machine


The Hivent Operations are the main contribution of this Master's thesis.

\begin{description}[labelindent=0.55em]
  \item[\textbf{3)}]
  \textbf{
    How can the model handle uncertainty and disagreement in history?
  }
\end{description}

% section results (end)


% ==============================================================================
\section{Problems} % (fold)
\label{sec:problems}

% section problems (end)



% ==============================================================================
\section{Prospect} % (fold)
\label{sec:prospect}


implementation of extensions
+ other shortcomings identified in evaluation
+ retrospective updates and backward changes
+ import of historical data from Wikipedia
+ integration of tool for semi-automatic extraction of historical maps

HistoGlobe would have the potential to be a good Historical Geographic Information System showing the history of countries in time and space.

If scholars in humanities use the Edit Mode of HistoGlobe to continuously improve the historical data in the system, animate historical countries and discuss historical events, then HistoGlobe would be a comprehensive historical world atlas as invisioned by its inventor. It would help teachers to explain  history to their teachers and students to understand the seemingly complex course of history

An implementation of these would yield a Historical Geographic Information System that is well suitable for visualizing and editing the history of countries on Earth.


has a great potential to teach, learn and understand processes in the past. A system that is able to tell \emph{what} happened and \emph{where} the historical event has influences on and \emph{when} the event happened happened might be tool to answer the most important of all questions:
\textbf{\emph{Why}} it happened?


finally to reason about if wars actually make sense.
If John Lennon is right, then this HGIS has come to its ultimate end: all Areas have unified. All the people living life in peace. You may say I am dreamer, but I am not the only. I hope someday you'll join us. And the world will be as one.

% section prospect (end)

% ==============================================================================

% chapter summary (end)