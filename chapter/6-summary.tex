%!TEX root = ../masters_thesis.tex

\chapter{Summary} % (fold)
\label{cha:summary}
has a great potential to teach, learn and understand processes in the past. A system that is able to tell \emph{what} happened and \emph{where} the historical event has influences on and \emph{when} the event happened happened might be tool to answer the most important of all questions:
\textbf{\emph{Why}} it happened?


finally to reason about if wars actually make sense.
If John Lennon is right, then this HGIS has come to its ultimate end: all Areas have unified. All the people living life in peace. You may say I am dreamer, but I am not the only. I hope someday you'll join us. And the world will be as one.

% ==============================================================================
\section{Results} % (fold)
\label{sec:results}

% section results (end)


\paragraph{Research Questions} % (fold)
\label{par:result_research_questions}

% paragraph result_research_questions (end)


% ==============================================================================
\section{Problems} % (fold)
\label{sec:problems}

% section problems (end)



% ==============================================================================
\section{Future Work} % (fold)
\label{sec:future_work}

step further: temporal GIS to narrative GIS

idea: explain history with spatial narratives
  geographically contextualize events and interactions
  organizing principle: time

extend the pure presentational purpose of st data to analytical purpose, e.g. where have most border changes take place in previous 200 years?

Another problem for historians is that they do not necessarily need a tool to better visualize existing knowledge (e.g. historical maps), but to generate new knowledge by analyzing spatio-temporal coherences or distributions in historical data. Spatio-temporal reasoning is still an open field and not easily possible with existing HGIS

\cite[p. 268]{knowles2008placing}, \cite[p. xii]{gregory2014toward}.
space-time premise by Gaddis 2002
  time and space equal importance
  event     what significantly has happend and by whom? (singularity!)
  process   how something has happened? (event+activity => trigger of process)
  change    driven by process
  spatiotemporal data defines all above three
  % \cite Gaddis 2002

extend area model to hierachies (country -> states -> counties/cities)

% section future_work (end)


% ==============================================================================

% chapter summary (end)