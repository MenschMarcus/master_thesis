%!TEX root = ../masters_thesis.tex

\chapter{Summary} % (fold)
\label{cha:summary}


% ==============================================================================
\section{Results} % (fold)
\label{sec:results}

% section results (end)


\paragraph{Research Questions} % (fold)
\label{par:result_research_questions}

% paragraph result_research_questions (end)


% ==============================================================================
\section{Problems} % (fold)
\label{sec:problems}

% section problems (end)



% ==============================================================================
\section{Future Work} % (fold)
\label{sec:future_work}

step further: temporal GIS to narrative GIS

idea: explain history with spatial narratives
  geographically contextualize events and interactions
  organizing principle: time

extend the pure presentational purpose of st data to analytical purpose, e.g. where have most border changes take place in previous 200 years?

% section future_work (end)


% ==============================================================================

% chapter summary (end)