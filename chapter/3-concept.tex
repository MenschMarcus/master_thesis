%!TEX root = ../thesis.tex

\section{Concept}
\label{sec:concept}

\subsection{Elements and Components of GIS}
\label{sub:elements_and_components_of_gis}

\begin{description}
    \item[Presentation]
    \item[Analysis]
    \item[Management]
    \item[Acquisition]
\end{description}

data model:
2D geometries
no self intersection
clean geometries
allow to be non-exhaustive (have holes e.g. enclaves, exclaves)
rules of simple polygon (look @ wikipedia)

feature topology table
points          ID, coordinates
polylines       ID, [points]        no direction, all splines order 1
polygons        ID, [polylines]
polypolygons    ID, [polygons]      account for islands
--> concept of ``multiaffilation''

interior border: discrete phenomenon
coastline: continuous phenomenon
-> combined for whole country border
-> have to deal with both

idea: strict connectivity -> topological network

modeling historical border changes


% section elements_and_components_of_gis (end)


\subsection{Evaluation}
\label{sub:evaluation}

\begin{description}
    \item[Effectiveness]
    \item[Efficiency]
    \item[Satisfaction]
\end{description}


\subsubsection{Evaluation Methods} % (fold)
\label{ssub:evaluation_methods}
% subsubsection evaluation_methods (end)

\subsubsection{Research Hypotheses} % (fold)
\label{ssub:research_hypotheses}

\begin{enumerate}
    \item name \\
    description
    \item name \\
    description
    \item name \\
    description
\end{enumerate}
% subsubsection research_hypotheses (end)

% subsection evaluation (end)

% section concept (end)

\vspace{2em}
transition to next chapter
