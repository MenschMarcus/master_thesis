%!TEX root = ../masters_thesis.tex

\chapter{Introduction} % (fold)
\label{cha:introduction}

\begin{quoteit}
\large
La Géographie n’est autre chose que l’Histoire dans l’espace, \\
de même que l’Histoire est la Géographie dans le temps. \\

Geography is nothing but History in space, \\
the same way as History is Geography over time.
\end{quoteit}
\hfill \textit{-- Élisée Reclus: ``L'Homme et la Terre'' (1908)}

long and aspiring text about human life, the question of where, when, what and how?


time and space are everywhere, highly related to our lives and objects we perceive
time
  personal
    time points of major life events, ...-> events, can trigger other events
    periods of studying, working, ... -> collection of events with similar characteristics
  world: every major issue has a time scale
    climate change (decades)
    climate tipping points (years) climate tipping points (years)
    economic meltdown (months)
    infectious diseases (weeks)
    disasters (days)
  -> time not easy to scale and to grasp
space
  location of major life events (static)
  travel routes (dynamic)
  not always easy to grasp (exact location of monument is simple, but exact location of problematic area with fremdenfeindlichen hintergrund or historic countries are hard to set)
  event locations are sometimes not related to their consequences (e.g. Conferences of Tehran or Casablanca (1943) discussed how to deal with Germany after planned victory in WWII)
motivation for spatio-temporal queries
  exploration of German history using historical maps of 1800, 1850, 1900 and 1950
    each map has both temporal and spatial information in it
    but how to tell a story with that?
    more realistically, maps from 1871, 1919, 1933, 1945 and 1949, because of major events (founding of German Reich, end of WWI, beginning of Nazi dictatorship, end of WWII, founding of two German nations
    -> for one country might be suitable
  exploration of European or history
    would need a world map for each year
    how to see what has changed? -> inefficient
    how to know what is important?
    is that a reasonable way of storing information if one information set is with a high probability almost the same as the time point before? -> redundancy
  key problem
    model of historic maps at time points (-> snapshot model)
    given information at time point t1 and t2
    How to know the status at time point t1 < tm < t2?
    -> It is impossible
  solution: away from snapshot based modeling of history to change-based modeling
    initial state ti, changes at point t1 and t2
    How to know the status at time point t1 < tm < t2?
    -> it is ti + changes at t1
    => definition of each time point in history

research object of this thesis
  change over time of space of countries
  history of countries, their names and their borders and their relationships to each other
  visualize these changes and edit them (interface)
  Web-based historical geographic information system (WHGIS)

research questions of the thesis
  How to design and implement WHGIS?
  How to create an interface not just to explore historical changes?
  How to deal with uncertainty and fuzziness in history?
  Can researchers actually interact with such a system?
  How to design an interface that matches the mental model of a DH user of editing changes over time?

study of existing approaches, techniques and projects
  GIS: acquisition, management, analysis and presentation of spatial information
  handling of the spatial domain: extension to HGIS
  some systems allow presentations, but have very difficult interfaces
  no system that allows editing historical borders in time and space

domain: evolution of countries in time and space
-> names and territories
-> modelling, visualizing and !!! editing !!!

% ==============================================================================
\section{Motivation} % (fold)
\label{sec:motivation}


research questions --> HGIS <-- development of system
  historians /                           ME
  geographers
(+) open source, direct manipulation, easy sharing and collaboration

% section motivation (end)


% ==============================================================================
\section{Research Questions} % (fold)
\label{sec:research_questions}


% section research_questions (end)


% ==============================================================================
\section{Overview} % (fold)
\label{sec:overview}

% section overview (end)


% ==============================================================================

% chapter introduction (end)