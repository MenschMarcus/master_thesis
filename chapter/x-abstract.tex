%!TEX root = ../masters_thesis.tex

\chapter*{Abstract} % (fold)
\label{cha:abstract}

Who we are is largely determined by where we come from. Our home country provides identification and shelter. It has a distinct history, but in the course of this history every country is subject to several changes. These can be caused by wars, revolutions or governmental decisions. If we want to understand why the world is the way it is today, then we would benefit from a visualization that shows the development of countries in time and space. This thesis aims to provide a foundation for a Historical Geographic Information System (HGIS) to explore and edit the history of countries.

The historical development of countries is driven by historical events that are called \emph{Hivents}. The Hivent model developed in this thesis is based on five fundamental operations that describe historical changes to countries: unification, incorporation, separation, secession and name change. A user interface is developed to visualize the history on a map including a timeline with the option to edit countries directly on the map. This is implemented in HistoGlobe, a web-based HGIS that serves as a historical world atlas.

A major problem of this undertaking is that almost everything we believe to ``know'' about our history is possibly uncertain. We do not really know how the world looked like 300 years ago and an unambiguous definition of what constitutes a ``country'' can not be found. If we compare history books in various parts of this world, it becomes apparent that there is more than just one side of each story. Therefore, the Hivent model is extended to support different kinds of uncertainty and disagreement, e.g.\ a contested territory, historical borders with low expected accuracy or different historical sources for the same Hivent.


% chapter abstract (end)
